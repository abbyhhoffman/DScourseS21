\documentclass{article}
\usepackage[utf8]{inputenc}

\title{DScourseS21 PS1}
\author{Abigail Hoffman }
\date{January 2021}

\usepackage{natbib}
\usepackage{graphicx}

\begin{document}

\maketitle

\section{Summary}
	Initially, I had no interest in economics. I was offered to take AP Macroeconomics in high school and decided that would be a waste of time. What would I ever need that material for in life? I pursued International Business and Accounting when I first began college. Ironically, I loved my fundamentals of economics classes more than the other Price college of business classes. Fast forward a few years and now as I senior I am almost finished with my MA/BA degree in economics with my minor in Spanish. I have never had an interest in data science or programming or how computer languages run, which I think influenced my decision to take this course. After being challenged by Econometrics and Statistics for Decision Making, I realized how invaluable the skill of programming and understanding how to use code can be in my future. These courses have been the first type of classes at OU that have forced me out of my comfort zone to work harder to achieve my goals. 

As of right now I do not have any ideas of what my project could be about. I do know that like my previous projects, something will come to fruition! 
My goals for this class is to remain \textit{coach-able}. I strive for perfection but when I am pushed out of my element it can be destructive. My goal is to be open minded and work towards accomplishing the elements of the class without fear of what grade I make. 

After graduation I will begin working for The Federal Reserve Bank of Kansas City in the Oklahoma City branch and will partake in their bank examiners program. I am still contemplating other options like law school and potentially data analyst jobs. I took this class so that I can expand my skills (build my resume) and potentially work for private sector companies as a data scientist/ analyst.


\section{Equation}
$a^2+b^2=c^2$

\end{document}